


\documentclass[]{article}
\usepackage{amssymb}


\title{Pre-signed redacted text signature verification via NIZKP}
\author{Noah Ostle}

\begin{document}

\maketitle

\newpage

\section{Design}
\textbf{Aim:} To be able to take a message $M$ signed by an original signer eg. Alice, redact it to $M'$, publish it, and prove that it was signed by Alice without revealing the redacted portion of the message.\\

\noindent \textbf{Design Goals:} Essentially, we want a \textit{non-interactive zero knowledge proof} that convinces a verifier;

\begin{itemize}
	\item There exists a message $M$ whose hash was signed by Alice.\
	\item The public part of the message $P$ concatenated with the redacted part of the message $w$ equals $M$.\\
\end{itemize}

\noindent Without revealing;
\begin{itemize}
	\item The redacted text, $w$\
	\item The hash/signature of $M$, as doing so would allow an attacker to bruteforce guess $w$ by making a guess $w'$, and comparing the hash of $P | w'$ with the hash of $M$.\\
\end{itemize}


\noindent \textbf{Scheme:}\\

Let $H\;:\; \{0,1\}^* \to \{0,1\}^{256}$ be a 256-bit cryptographic hash function, eg. SHA-256.\\

Let $V\;:\;PK_a \times \{0,1\}^{256}\times \sigma_a(M) \to \{0,1\}$ be a signature verification algorithm, where $PK_a$ is Alice's public key, and $\sigma_a(M)$ is Alice's signature on message $M$. Eg. a RSA-PSS or PKCS signature.\\

First, we let Alice's original message be $M$, and from it, decide what we want to publish; $P$, and what we want to redact; $w$.\\For example we may want to prove we solved a puzzle, but not reveal our answer;

\begin{quote}
$M$ = Congratulations! Your answer 'Sup3R53cR3t3301!' was correct.
\end{quote}

\begin{quote}
	$P_1$ = Congratulations! Your answer,    $\;\;\;P_2$ = was correct. 
\end{quote}

\begin{quote}
	$w$ = 'Sup3R53cR3t3301!'
\end{quote}

\noindent \textit{Note: you can continue to split up the text into multiple $P_n$'s and $w_i$'s, but for the purpose of this explanation we will just write $P,w$ without indices st. $M=P||w$ to save space.}\\

If we imagine $M$, and subsequently $P$ and $w$ as byte strings; $P,w \in \{0,1\}^*$, then we can naturally define the hash of the message $h_M$ as;

$$
h_M := H(M) = H(P||w)
$$

\noindent We can now generate criteria for a non-interactive zero knowledge proof $\pi$:

\begin{quote}

	Public inputs: $PK_a,\; P$\\
	Witness: $(w, \; h_M, \; \sigma_a(M))$\\
	
	Relation: $R( (PK_a,\; P), \; (w, \; h_M, \; \sigma_a(M))) = 1 \iff\\$
	\hspace*{1.55cm}$(h_M=H(P||w)) \; \land \; (V(PK_a, \; h_M,\;  \sigma_a(M))=1)$\\
	
	\textit{ie. The relation is true when the hash of the public message and the redacted portion concatenated is equal to the hash of M, and there exists a valid signature by Alice for the hash of M.}\\
	
	$\therefore$ A verifier should accept the proof iff:
	$$\exists (w, \; h_M, \; \sigma_a(M)) \;:\; (h_M=H(P||w)) \; \land \; (V(PK_a, \; h_M,\;  \sigma_a(M))=1)$$

\end{quote}

As you can see, the relation $R$ relies on the hashing function $H$, and the signature verification function $V$.\\In theory, we could choose any hashing algorithm and signature algorithm, but for this example we will use RSA PKCS \#1 v 1.5 and SHA256. \\If we can prove that these algorithms can be performed on a fixed input in polynomial time, then we can construct a satisfiable boolean circuit for use with ZK-SNARK machinery.\\ Firstly, we will take the example of a textbook RSA signature. RSA signature verification involves computing $\sigma^e \;\texttt{mod}\;n$ using modular exponentiation as specified in PKCS \#1 v2.2 (RFC 8017) \cite{pkcs}. The dominant cost is modular exponentiation on a k-bit modulus. With naive integer arithmetic, modular multiplication is $O(k^2)$ bit operations and square-and-multiply exponentiation uses $O(log\;e)$ such multiplications. Thus, when the public exponent e is fixed or bounded (as in best practice), RSA verification runs in $O(k^2)$; Additionally, SHA-256 runs in $O(n)$ as per NIST FIPS PUB 180-4 \cite{nist} \textbf{TODO: describe the ZKSNARK circuit design}





\newpage
\bibliographystyle{IEEEtran}
\bibliography{IEEEabrv,refs.bib}
\newpage
\end{document}
